\documentclass{article}
\usepackage{graphicx} % Required for inserting images

\title{TF-IDF TRABALHO - Enzo Ura}
\author{Enzo Ura}
\date{November 2025}

\maketitle

\section{Introdução}

\documentclass[12pt,a4paper]{article}

\usepackage[utf8]{inputenc}
\usepackage[brazil]{babel}
\usepackage{geometry}
\usepackage{graphicx}
\usepackage{listings}
\usepackage{color}
\usepackage{amsmath}

\geometry{margin=2.5cm}

\definecolor{lightgray}{gray}{0.95}

\lstset{
  backgroundcolor=\color{lightgray},
  basicstyle=\ttfamily\footnotesize,
  frame=single,
  breaklines=true,
  postbreak=\mbox{\textcolor{red}{$\hookrightarrow$}\space}
}

\title{Sistema de Recomendação Odontológica com TF-IDF}
\author{Aluno: \textbf{Enzo Shinji Sugano Ura}}
\date{\today}

\begin{document}

\maketitle

\section{Introdução}
O  trabalho tem como objetivo desenvolver um sistema de recomendação de procedimentos odontológicos utilizando a técnica de \textbf{TF-IDF} (Term Frequency–Inverse Document Frequency) e a métrica de \textbf{similaridade do cosseno}. 

A proposta é permitir que o usuário digite um procedimento odontológico ou uma breve descrição, e o sistema retorne os tratamentos mais semelhantes com base em um banco de dados textual. A escolha do tema visa demonstrar a aplicação prática de técnicas de mineração de texto na área da saúde.

\section{Descrição do Dataset}
O conjunto de dados é composto por 10 procedimentos odontológicos, com suas respectivas descrições resumidas:

\begin{itemize}
  \item \textbf{Limpeza dental} — remoção de placa bacteriana e tártaro, polimento e aplicação de flúor.
  \item \textbf{Restauração dental} — tratamento para reparar dentes danificados por cáries ou fraturas.
  \item \textbf{Extração dental} — remoção de dentes comprometidos por cáries profundas ou infecção.
  \item \textbf{Clareamento dental} — procedimento estético para clarear o tom dos dentes.
  \item \textbf{Canal (endodontia)} — tratamento para remover a polpa dentária infectada e selar o dente.
  \item \textbf{Implante dentário} — colocação de pino de titânio para substituir dentes ausentes.
  \item \textbf{Aparelho ortodôntico} — dispositivo fixo ou móvel para corrigir o alinhamento dos dentes.
  \item \textbf{Profilaxia infantil} — limpeza preventiva em crianças para evitar cáries e gengivite.
  \item \textbf{Raspagem periodontal} — remoção profunda de tártaro abaixo da gengiva.
  \item \textbf{Prótese dentária} — substituição de dentes perdidos por próteses fixas ou removíveis.
\end{itemize}

\section{Metodologia}
O sistema foi implementado em \textbf{Python} utilizando a biblioteca \texttt{scikit-learn}. Cada descrição textual é convertida em um vetor numérico com base no modelo \textbf{TF-IDF}.

\subsection{Cálculo do TF-IDF}
O TF-IDF mede a relevância de uma palavra dentro de um documento e em relação ao conjunto total de documentos.  
A fórmula geral é:
/n
/n
\[
TFIDF(t, d) = TF(t, d) \times IDF(t)
\]

onde:

\begin{itemize}
  \item $TF(t, d)$ é a frequência do termo $t$ no documento $d$.
  \item $IDF(t) = \log(\frac{N}{DF(t)})$, sendo $N$ o número total de documentos e $DF(t)$ a quantidade de documentos que contêm o termo $t$.
\end{itemize}

Com isso, palavras comuns (como “de”, “o”, “para”) têm peso reduzido, enquanto termos específicos (como “tártaro”, “clareamento”, “implante”) ganham mais importância.

\subsection{Similaridade do Cosseno e o Ângulo entre Vetores}
Cada documento (descrição de procedimento) é representado como um vetor de pesos TF-IDF.  
Para medir o quanto dois procedimentos são semelhantes, calcula-se o \textbf{cosseno do ângulo} entre esses vetores:

\[
\text{Similaridade}(A, B) = \frac{A \cdot B}{\|A\| \times \|B\|}
\]

Quando o ângulo é pequeno (vetores próximos), a similaridade é alta (valor próximo de 1).  
Quando o ângulo é grande (vetores distantes), a similaridade é baixa (valor próximo de 0).

\section{Análise das Semelhanças e dos Ângulos}
A análise dos ângulos mostra como o modelo identifica relações semânticas entre os procedimentos.  
Por exemplo:

\begin{itemize}
  \item \textbf{Limpeza dental} e \textbf{Profilaxia infantil} — ambas envolvem remoção de placa e prevenção, portanto apresentam alta similaridade (ângulo pequeno, cerca de 25° a 30°).
  \item \textbf{Raspagem periodontal} e \textbf{Limpeza dental} — possuem objetivos semelhantes, mas uma é mais profunda. Similaridade moderada (ângulo de 40° a 50°).
  \item \textbf{Clareamento dental} e \textbf{Implante dentário} — tratam de áreas distintas (estética e cirurgia), logo o ângulo entre os vetores é grande (próximo de 80°), indicando baixa similaridade.
  \item \textbf{Canal} e \textbf{Extração dental} — embora sejam tratamentos diferentes, ambos envolvem dentes danificados e processos de remoção de tecido, com similaridade média (ângulo de 45° a 55°).
\end{itemize}

Essas comparações mostram que o TF-IDF, mesmo sendo um método simples, consegue capturar padrões relevantes de contexto.  
O “ângulo” funciona como uma medida geométrica de semelhança textual: quanto mais próximas as descrições, mais “paralelos” são os vetores no espaço vetorial.

\section{Interação com o Usuário}
O programa é interativo: o usuário informa o nome ou uma descrição do procedimento e o sistema retorna os três mais semelhantes.  
Para encerrar a execução, basta digitar \texttt{sair}.

\section{Código em Python}
\begin{lstlisting}[language=Python]
from sklearn.feature_extraction.text import TfidfVectorizer
from sklearn.metrics.pairwise import cosine_similarity

procedimentos = {
    "Limpeza dental": "remoção de placa bacteriana e tártaro, polimento e flúor.",
    "Restauração dental": "tratamento para reparar dentes danificados.",
    "Extração dental": "remoção de dentes comprometidos por cáries profundas.",
    "Clareamento dental": "clarear o tom dos dentes com agentes clareadores.",
    "Canal (endodontia)": "remoção da polpa dentária infectada e selamento.",
    "Implante dentário": "colocação de pino de titânio para substituir dentes ausentes.",
    "Aparelho ortodôntico": "uso de dispositivo fixo ou móvel para corrigir dentes.",
    "Profilaxia infantil": "limpeza preventiva em crianças para evitar cáries.",
    "Raspagem periodontal": "remoção profunda de tártaro abaixo da gengiva.",
    "Prótese dentária": "substituição de dentes perdidos por próteses."
}

nomes = list(procedimentos.keys())
descricoes = list(procedimentos.values())

vetorizador = TfidfVectorizer()
matriz_tfidf = vetorizador.fit_transform(descricoes)

print("=== Sistema de Recomendação Odontológica ===")
print("Digite um procedimento (ou 'sair' para encerrar)")

while True:
    entrada = input("Procedimento: ").strip().lower()
    if entrada == "sair":
        print("Encerrando o sistema...")
        break

    entrada_tfidf = vetorizador.transform([entrada])
    similaridades = cosine_similarity(entrada_tfidf, matriz_tfidf)[0]
    indices_ordenados = similaridades.argsort()[::-1]

    print("\nMais semelhantes:")
    for i in indices_ordenados[:3]:
        print(f"- {nomes[i]} ({similaridades[i]:.2f})")
\end{lstlisting}

\section{Conclusão}
O uso do TF-IDF aliado à similaridade do cosseno mostrou-se eficaz na criação de um sistema de recomendação textual simples e funcional.  
Mesmo com um conjunto de dados pequeno, a técnica conseguiu identificar relações reais entre os procedimentos odontológicos com base na linguagem natural.

A interpretação dos ângulos entre vetores reforça a importância do modelo: ângulos pequenos indicam forte correlação entre termos, enquanto ângulos grandes revelam diferenças semânticas significativas.

Como aprimoramentos futuros, pode-se incluir:
\begin{itemize}
  \item Análise semântica com embeddings (Word2Vec, BERT);
  \item Expansão do dataset com descrições mais detalhadas;
  \item Interface gráfica para facilitar a interação do usuário.
\end{itemize}

\end{document}
