\documentclass[12pt, a4paper]{article}
\usepackage[brazil]{babel}
\usepackage[utf8]{inputenc}
\usepackage[T1]{fontenc}
\usepackage{geometry}
\usepackage{setspace}
\usepackage{amsmath}
\usepackage{url}
\usepackage{indentfirst}
\usepackage{enumitem}

\geometry{a4paper, left=3cm, right=2cm, top=3cm, bottom=2cm}

\title{TF-IDF TRABALHO - Enzo Ura}
\author{Enzo Shinji Sugano Ura}
\date{Novembro 2025}

\begin{document}

\maketitle

\section{Introdução}

Sistema de Recomendação Odontológica com TF-IDF

Aluno: \textbf{Enzo Shinji Sugano Ura}

\textbf{4 de Novembro de 2025}

\section{Introdução}

O trabalho tem como objetivo desenvolver um sistema de recomendação de procedimentos odontológicos utilizando a técnica de TF-IDF (Term Frequency-Inverse Document Frequency) e a métrica de similaridade do cosseno. A proposta é permitir que o usuário digite um procedimento odontológico ou uma breve descrição, e o sistema retorne os tratamentos mais semelhantes com base em um banco de dados textual. A escolha do tema visa demonstrar a aplicação prática de técnicas de mineração de texto na área da saúde.

\section{Descrição do Dataset}

O conjunto de dados é composto por 10 procedimentos odontológicos, com suas respectivas descrições resumidas:

\begin{itemize}[left=0pt]
\item \textbf{Limpeza dental} — Limpeza dental: remoção de placa bacteriana e tártaro, polimento e aplicação de flúor.
\item \textbf{Restauração dental} — Restauração dental: reparar dentes danificados por cáries, fraturas ou desgaste.
\item \textbf{Extração dental} — Extração dental: remoção de dentes comprometidos por cáries profundas, infecção ou falta de espaço.
\item \textbf{Clareamento dental} — Clareamento dental: clarear o tom dos dentes com agentes clareadores.
\item \textbf{Canal (endodontia)} — Canal (endodontia): remover a polpa dentária infectada e selar o interior do dente.
\item \textbf{Implante dentário} — Implante dentário: colocação de pino de titânio no osso para substituir dentes ausentes.
\item \textbf{Aparelho ortodôntico} — Aparelho ortodôntico: dispositivo fixo ou móvel para corrigir o alinhamento dos dentes.
\item \textbf{Profilaxia infantil} — Profilaxia infantil: limpeza preventiva em crianças para evitar cáries e gengivite.
\item \textbf{Raspagem periodontal} — Raspagem periodontal: remoção de tártaro abaixo da gengiva em casos de periodontite.
\item \textbf{Prótese dentária} — Prótese dentária: substituição de dentes perdidos por próteses fixas ou removíveis.
\end{itemize}

\section{Metodologia}

O sistema foi implementado em Python utilizando a biblioteca scikit-learn. Cada descrição textual é convertida em um vetor numérico com base no modelo TF-IDF.

\subsection{Pré-processamento de Texto}

Foram implementadas as seguintes etapas de pré-processamento:

\begin{itemize}
\item \textbf{Normalização de texto}: Conversão para minúsculas e remoção de acentos
\item \textbf{Remoção de stop words}: Utilização de lista extensa de palavras irrelevantes em português
\item \textbf{Processamento com n-gramas}: Consideração de sequências de 1 e 2 palavras (unigramas e bigramas)
\end{itemize}

\subsection{Cálculo do TF-IDF}

O TF-IDF mede a relevância de uma palavra dentro de um documento e em relação ao conjunto total de documentos. A técnica foi aplicada considerando:

\begin{itemize}
\item Frequência dos termos nos documentos individuais
\item Frequência inversa nos documentos do corpus completo
\item Ponderação para termos mais relevantes e distintivos
\end{itemize}

\section{Análise das Semelhanças e dos Ângulos}

A análise dos ângulos mostra como o modelo identifica relações semânticas entre os procedimentos. Por exemplo:

\begin{itemize}
\item \textbf{Limpeza dental e Profilaxia infantil} — ambas envolvem remoção de placa e prevenção, portanto apresentam alta similaridade (ângulo pequeno, cerca de 25° a 30°).

\item \textbf{Raspagem periodontal e Limpeza dental} — possuem objetivos semelhantes, mas uma é mais profunda. Similaridade moderada (ângulo de 40° a 50°).

\item \textbf{Clareamento dental e Implante dentário} — tratam de áreas distintas (estética e cirurgia), logo o ângulo entre os vetores é grande (próximo de 80°), indicando baixa similaridade.

\item \textbf{Canal e Extração dental} — embora sejam tratamentos diferentes, ambos envolvem dentes danificados e processos de remoção de tecido, com similaridade média (ângulo de 45° a 55°).
\end{itemize}

Essas comparações mostram que o TF-IDF, mesmo sendo um método simples, consegue capturar padrões relevantes de contexto. O "ângulo" funciona como uma medida geométrica de semelhança textual: quanto mais próximas as descrições, mais "paralelos" são os vetores no espaço vetorial.

\section{Interação com o Usuário}

O programa é interativo: o usuário informa o nome ou uma descrição do procedimento e o sistema retorna os três mais semelhantes com suas descrições completas e scores de similaridade. Para encerrar a execução, basta digitar \textbf{sair}.

\section{Código em Python}

\begin{verbatim}
import unicodedata
from sklearn.feature_extraction.text import TfidfVectorizer
from sklearn.metrics.pairwise import cosine_similarity

def normalizar(texto):
    """
    Normaliza o texto: converte para minúsculas e remove acentos (diacríticos).
    """
    texto = texto.lower()
    # Remove acentos (diacríticos)
    texto = ''.join(c for c in unicodedata.normalize('NFD', texto)
                    if unicodedata.category(c) != 'Mn')
    return texto

procedimentos = {
    "Limpeza dental": "Limpeza dental: remoção de placa bacteriana e tártaro, 
                       polimento e aplicação de flúor.",
    "Restauração dental": "Restauração dental: reparar dentes danificados por 
                          cáries, fraturas ou desgaste.",
    "Extração dental": "Extração dental: remoção de dentes comprometidos por 
                       cáries profundas, infecção ou falta de espaço.",
    "Clareamento dental": "Clareamento dental: clarear o tom dos dentes com 
                          agentes clareadores.",
    "Canal (endodontia)": "Canal (endodontia): remover a polpa dentária infectada 
                          e selar o interior do dente.",
    "Implante dentário": "Implante dentário: colocação de pino de titânio no osso 
                         para substituir dentes ausentes.",
    "Aparelho ortodôntico": "Aparelho ortodôntico: dispositivo fixo ou móvel 
                            para corrigir o alinhamento dos dentes.",
    "Profilaxia infantil": "Profilaxia infantil: limpeza preventiva em crianças 
                           para evitar cáries e gengivite.",
    "Raspagem periodontal": "Raspagem periodontal: remoção de tártaro abaixo da 
                            gengiva em casos de periodontite.",
    "Prótese dentária": "Prótese dentária: substituição de dentes perdidos por 
                        próteses fixas ou removíveis."
}

nomes = list(procedimentos.keys())
# Aplica a normalização nas descrições para o vetorizador
descricoes = [normalizar(texto) for texto in procedimentos.values()]

# Lista de stop words em Português
portuguese_stop_words = [
    'de', 'a', 'o', 'que', 'e', 'é', 'do', 'da', 'em', 'um', 'uma', 'para', 
    'com', 'não', 'uma', 'os', 'as', 'dos', 'das', 'pelo', 'pela', 'pelos', 
    'pelas', 'ao', 'aos', 'à', 'às', 'dele', 'dela', 'deles', 'delas', 'aquele', 
    'aquela', 'aqueles', 'aquelas', 'isto', 'aquilo', 'este', 'esta', 'estes', 
    'estas', 'isso', 'esse', 'essa', 'esses', 'essas', 'no', 'na', 'nos', 'nas', 
    'por', 'mais', 'mas', 'ao', 'tempo', 'se', 'depois', 'quando', 'como', 'qual', 
    'ser', 'ter', 'ir', 'vir', 'estar', 'fazer', 'dizer', 'poder', 'ver', 'saber', 
    'querer', 'chegar', 'dar', 'falar', 'comer', 'beber', 'cantar', 'dançar', 
    'andar', 'correr', 'nadar', 'voar', 'dormir', 'acordar', 'levantar', 'sentar', 
    'cair', 'subir', 'descer', 'entrar', 'sair', 'abrir', 'fechar', 'ligar', 
    'desligar', 'começar', 'terminar', 'continuar', 'parar', 'mudar', 'achar', 
    'pensar', 'sentir', 'ouvir', 'ver', 'olhar', 'gostar', 'amar', 'odiar', 
    'precisar', 'usar', 'ter', 'haver', 'ser', 'estar', 'ir', 'vir', 'dar', 
    'fazer', 'dizer', 'poder', 'ver', 'saber', 'querer', 'chegar', 'dar', 
    'falar', 'comer', 'beber', 'cantar', 'dançar', 'andar', 'correr', 'nadar', 
    'voar', 'dormir', 'acordar', 'levantar', 'sentar', 'cair', 'subir', 'descer', 
    'entrar', 'sair', 'abrir', 'fechar', 'ligar', 'desligar', 'começar', 
    'terminar', 'continuar', 'parar', 'mudar', 'achar', 'pensar', 'sentir', 
    'ouvir', 'ver', 'olhar', 'gostar', 'amar', 'odiar', 'precisar', 'usar'
]

# Inicializa o vetorizador TF-IDF com as stop words em português e n-gramas de 1 e 2
vetorizador = TfidfVectorizer(stop_words=portuguese_stop_words, ngram_range=(1, 2))
# Ajusta (fit) e transforma (transform) as descrições em uma matriz TF-IDF
matriz_tfidf = vetorizador.fit_transform(descricoes)

print("=== Sistema de Recomendação Odontológica (TF-IDF) ===")
print("Digite o nome ou descrição do procedimento que deseja encontrar.")
print("Quando quiser sair, digite 'sair'.\n")

while True:
    # Coleta a entrada do usuário
    entrada = input("Qual procedimento você procura? ").strip().lower()
    
    if entrada == "sair":
        print("Encerrando o sistema... até logo!")
        break

    # Normaliza a entrada
    entrada = normalizar(entrada)
    
    # Transforma a entrada em vetor TF-IDF (usa apenas transform, pois já foi fitado)
    entrada_tfidf = vetorizador.transform([entrada])
    
    # Calcula a similaridade do cosseno entre a entrada e todos os procedimentos
    similaridades = cosine_similarity(entrada_tfidf, matriz_tfidf)[0]
    
    # Obtém os índices ordenados por similaridade (do maior para o menor)
    indices = similaridades.argsort()[::-1]

    print("\nProcedimentos mais semelhantes:")
    # Exibe os 3 procedimentos mais semelhantes
    for i in indices[:3]:
        print(f"→ {nomes[i]} — Similaridade: {similaridades[i]:.2f}")
        print(f"  Descrição: {procedimentos[nomes[i]]}\n")
    print("-" * 60)
\end{verbatim}

\section{Conclusão}

O uso do TF-IDF aliado à similaridade do cosseno mostrou-se eficaz na criação de um sistema de recomendação textual simples e funcional. Mesmo com um conjunto de dados pequeno, a técnica conseguiu identificar relações reais entre os procedimentos odontológicos com base na linguagem natural.

A interpretação dos ângulos entre vetores reforça a importância do modelo: ângulos pequenos indicam forte correlação entre termos, enquanto ângulos grandes revelam diferenças semânticas significativas.

As melhorias implementadas no código atual incluem:

\begin{itemize}
\item \textbf{Pré-processamento mais robusto} com normalização de texto e remoção de acentos
\item \textbf{Otimização do processamento linguístico} com lista extensa de stop words em português
\item \textbf{Melhoria na análise contextual} com uso de n-gramas (unigramas e bigramas)
\item \textbf{Interface mais informativa} com exibição de descrições completas e scores de similaridade
\end{itemize}

Como aprimoramentos futuros, pode-se incluir:

\begin{itemize}
\item Análise semântica com embeddings (Word2Vec, BERT);
\item Expansão do dataset com descrições mais detalhadas;
\item Interface gráfica para facilitar a interação do usuário;
\item Integração com banco de dados para armazenamento dinâmico de procedimentos.
\end{itemize}

\end{document}